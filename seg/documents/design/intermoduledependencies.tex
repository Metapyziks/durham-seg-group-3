\newcolumntype{C}[1]{>{\Centering}p{#1}}
\def\TabS#1#2{\small\tabular{#1}\rule[-1.5mm]{0pt}{5mm}{#2}
    \rule[-2mm]{0pt}{2mm}\endtabular}
\def\TabA#1#2#3{\small\tabular{#1}\rule[-1.5mm]{0pt}{5mm}\textbf{#2}\\
    \hline\rule[0mm]{0pt}{4mm}#3
    \rule[-2mm]{0pt}{2mm}\endtabular}
\def\TabB#1#2#3{\small\tabular{c}\rule[-1.5mm]{0pt}{5mm}
    \textbf{{\tiny$\ll$}#1{\tiny$\gg$}}\\
    \hline\rule[0mm]{0pt}{4mm}\tabular{ll}#2\endtabular \\
    \hline\rule[0mm]{0pt}{4mm}\tabular{ll}#3\endtabular
    \rule[-2mm]{0pt}{2mm}\endtabular}
\seticonparams{entity}{
    shadow=false,
    fillcolor=blue!10,
    fillstyle=solid,
    framesep=0pt}

\subsubsection{Architecture Overview}
It was apparent from the briefing and requirements that the most suitable architecture for the system would be a client-server model. There will be a central server that stores persistent system related data, and performs the majority of the application logic. This server services requests from many client applications that handle user interaction and present information through a graphical user interface. A requirement of the application was also to provide a website as a second source of user interaction, which will be implemented using the central server.

The server and client may be conceptually divided into several modules. For the server, there is the \emph{Database Module} that manages reading from and modifying the persistent application data, the \emph{Logic Module} that handles game and system calculations, and the \emph{Web Interface} group which contains the \emph{API Module} for interacting with client applications, and the \emph{Website Module} for servicing the website HTTP requests. The client will be constructed to include a \emph{Request Module} for contacting the server and interpreting any responses, a \emph{Logic Module} that is aware of relevant aspects of the game state and performs client-relevant calculations, a \emph{Geolocation Module} for interaction with the Google APIs and the GPS sensor, and a \emph{Window Module} for the graphical user interface.

\begin{figure}[h!]
    \centering
    \begin{tabular}{c}
    \entity{server}[\TabA{c}{Centralized Server}{~\\[-3mm]
        \tabular{c}~\\[-10mm]
            \entity{database}[\TabS{c}{Database Module}]\\[8mm]
            \entity{serverlogic}[\TabS{c}{Logic Module}]\\[-3mm]
        \endtabular
        \hspace{5mm}
        \entity{web}[\TabA{c}{Web Interface}{~\\[-3mm]
            \entity{api}[\TabS{c}{API Module}]\\[2mm]
            \entity{website}[\TabS{c}{Website Module}]\\[-3mm] 
        }]\\[-3mm]
    }]
    \hspace{5mm}
    \entity{app}[\TabA{c}{Client Android Application}{~\\[3mm]
        \hspace{-5mm}
        \tabular{cc}
        \tabular{c}~\\[-10mm]
            \entity{request}[\TabS{c}{Request Module}]\\[8mm]
            \entity{applogic}[\TabS{c}{Logic Module}]\\[-3mm]
        \endtabular
        \hspace{2mm}
        \tabular{c}~\\[-10mm]
            \entity{window}[\TabS{c}{Window Module}]\\[8mm]
            \entity{geo}[\TabS{c}{Geolocation Module}]\\[-3mm]
        \endtabular\hspace{-3mm}
        \endtabular
        \\[3mm]
    }]
    \end{tabular}
    {\small
        \ncline[arrowscale=1.5]{<->}{database}{serverlogic}
        \ncline[arrowscale=1.5]{<->}{web}{serverlogic}
        \ncline[arrowscale=1.5]{->}{database}{web}
        \ncline[arrowscale=1.5]{<->}{api}{request}
        \ncline[arrowscale=1.5]{<->}{request}{applogic}
        \ncline[arrowscale=1.5]{<->}{window}{applogic}
        \ncline[arrowscale=1.5]{->}{request}{window}
        \ncline[arrowscale=1.5]{->}{geo}{applogic}
    }
    \caption{Data flow of the top level of the system, showing the application and server as clearly separated entities with internal modules that encapsulate distinct functionalities of the system.}
\end{figure}

As \emph{Figure 1} shows, the server and client have a similar overall structure. Both have a main contained logic processing module, a modifiable data source (the \emph{Database Module} for the server and the \emph{Request Module} for the client), and an interface (the \emph{API Module} and \emph{Website Module} for the server, and \emph{Window Module} for the client). This compartmentalisation of processes is intended to make both the server and client subsystems more expandable and maintainable while features are being implemented and testing performed. The interaction between modules is restricted to a small and manageable set of interfaces to help reduce the internal complexity of the system.

\subsubsection{Server Module Descriptions}
\paragraph{Database Module}
This module will abstract away interaction with the DMS (Database Management System), making it possible to easily replace the DMS used to support different platforms. For the Windows operating system the server will be running on top of the .NET CLR (Common Language Runtime), and will therefore have access to Microsoft's SQL Compact Edition Server. On Unix systems, the server will be using the Mono CLR implementation, and will therefore have to use an alternative DMS such as SQLite. As well as supporting different DMS connections and SQL dialects depending on the host system, the Database Module will also provide a simple interface to common SQL operations through the use of code generation. This will reduce the amount of errors produced through the use of poorly constructed SQL statements by delegating the validation task to the debugging facilities provided by the IDE (Integrated Development Environment) used while developing the server.

\paragraph{Logic Module}
The Logic Module will contain all critical game related calculations.