\section{Hardware and Software}
	The main product shall take the form of an ``Android app'' - a self 
	contained program running on an Android mobile device. This is what the 
	user will directly interact with. Our product will also require a central 
	server in order to function. The Android app shall be able to transfer 
	information to and from this sever.
	\subsection{Android Application}
		\subsubsection{Hardware}
			The device provided for testing the Android app is a HTC Desire C, 
			and the application should perform well on similar devices. The 
			relevant hardware capabilities of the Desire C are listed
			below\cite{htcdesirex}:

			\begin{itemize}
				\item Screen Resolution: 320 x 480 pixels
				\item Processor Speed: 600MHz
				\item Memory: 512MB
				\item SD Card Storage: 4GB +
				\item Battery Life: 10 - 20 hours of active use
				\item Internal GPS Antenna
			\end{itemize}

			\noindent
			The app is unlikely to require more capable hardware than what is 
			provided by the device because any intensive processing of data 
			can be offset to the central server. The most that the device will 
			need to handle is some trivial real-time graphical operations, and 
			for that the hardware is more than adequate. Memory should not be 
			a concern because the device has a relatively large amount for a 
			hand-held device, and this product will not require excessive 
			amounts of data to be stored. The app should not exhaust the 
			battery supply of the device to a prohibitive extent, and some 
			effort should be made during development to limit the usage of 
			battery draining resources. The product requires the ability to 
			find its current position using a GPS system, which is an ability 
			of the testing device.
		\subsubsection{Software}
			The application will be written in the programming language Java 
			using the development environment Eclipse, and will target the 
			Android operating system version 4.0 as a minimum as specified by 
			the client.

			Java will be used because it is the language the application 
			developers are most proficient in, and Eclipse was chosen for its 
			superior support for developing Android applications.
	\subsection{Central Server}
		\subsubsection{Hardware}
			The central server program will reside on a conventional computer, 
			and because this one machine will handle all client requests it 
			will need much more advanced hardware than the mobile devices. 
			This server shall have internet connectivity in order to 
			communicate with the Android app clients. The server will also 
			need to be almost constantly active, with downtime only occurring 
			at times of the day when few clients will want to connect. The 
			machine running the server requires a large upload bandwidth in 
			order to service many clients in a small time frame, and this 
			should be complemented by a high enough CPU and memory access 
			speed to reduce the processing time of client requests. A lower 
			bandwidth can be compensated for by reducing the size of data 
			being sent to clients. The server should have enough free storage 
			space to allow for expansion of the product to cover more area of 
			the world.
		\subsubsection{Software}
			The server will be written in the programming language C$\sharp$, 
			and built on the .NET Framework version 4.5 using the Visual 
			Studio 2012 development environment. C$\sharp$ was chosen for its 
			superior performance relative to Java, and because it is the 
			language the primary server developer is most experienced with. 
