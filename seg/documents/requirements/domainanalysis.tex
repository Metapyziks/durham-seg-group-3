\section{Domain Analysis}
	\subsection{Technical Analysis}
		The product will include a user account system, where users can create 
		accounts which store personal information, and can log in to their 
		account to access resources. Implementing this system will involve 
		solving several security and design problems, but thankfully the user 
		account model has been tried and tested in thousands of online 
		applications.

		Almost all websites with account registration require the user to 
		validate their email address in order to complete the account creation 
		process. This is generally implemented using an activation email sent 
		to the address they gave while registering. The user must either use 
		an identification code inside the email or click a link in the message 
		to activate their account. This will prove that the user has access to 
		that email address. Forcing the user to validate an email address can 
		help deter users creating many accounts because each account needs a 
		unique email address assigned to it. This can also counter the problem 
		of spam accounts being created by an automated system.

		Another problem is account security, and making sure a malicious user 
		cannot gain access to another user's account. One aspect of this is 
		requiring users to register with more complex passwords. A few 
		websites will not accept passwords which do not conform to certain 
		constraints, such as having at least one digit and a mixture of upper 
		and lower case. This prevents users from using simple and easy to 
		guess passwords. However, even the most complex and hard to guess 
		password is useless if an adversary can intercept it when being sent 
		to the server. To protect against this, passwords are usually hashed 
		with a hashing function that is very difficult or impossible to reverse
		. The server will never see the user's actual password, only a hash of 
		it. This has the added benefit of protecting every users password if 
		the server database is compromised. While it is true that if the 
		password is intercepted the adversary can use it to authenticate 
		themselves into that particular service, because the password is 
		hashed they can not find the original password and use it with other 
		services. This therefore protects users which use the same password 
		for many websites.

		A user's authentication credentials are more likely to be intercepted 
		if they need to send them frequently. This is conventionally prevented 
		by implementing authentication sessions, where the user only needs to 
		provide their credentials once per session. The server then creates a 
		hash code which identifies the session, and this code is sent to the 
		client. For subsequent requests which require authentication the 
		client only needs to send the session hash. The session usually 
		expires after a certain amount of time after it was initiated, or 
		since the last action during that session was made. To add even more 
		security, a session may be associated with the IP address of the 
		client which requested its creation. If an adversary were to listen in 
		and obtain the session hash, they would be unable to use it to 
		authenticate because their IP would differ from the legitimate user.

		The Data Protection Act (1998) requires that organisations holding 
		personal information must remove that information when the subject of 
		the data requests it\cite{dpa1998}. Usually this is complied with by 
		allowing the user to delete their account through an automated 
		process, without any intervention from an administrator.
