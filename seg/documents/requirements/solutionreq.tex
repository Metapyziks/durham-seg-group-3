\renewcommand{\arraystretch}{1.2}
\def\feedbackspace{0}
\newcommand{\funcreq}[7]
{
	\subsubsection{#1}
	\begin{tabular}{|>{\raggedleft\arraybackslash}p{3cm}|p{12cm}|}
		\hline
		\textbf{Type} &
			Functional
		\\
		\textbf{Description} & #2 \\
		\textbf{Priority} &
			\ifnum#3=1 High \else\ifnum#3=2 Medium \else Low \fi\fi
		\\ 
		\ifnum\pdfstrcmp{None}{#4}=0 \else
			\textbf{Pre-conditions} & #4 \\
		\fi
		\ifnum\pdfstrcmp{N/A}{#5}=0 \else
			\textbf{Input} & #5 \\ 
		\fi
		\ifnum\pdfstrcmp{N/A}{#6}=0 \else
			\textbf{Operations} & #6 \\
		\fi 
		\textbf{Expected Results} & #7 \\
		\hline
		\if\feedbackspace1
			\hline
			\textbf{Pass / Fail} &
				~
			\\
			\textbf{Remarks} &
				~\newline~\newline~
			\\ \hline
		\fi
	\end{tabular}
}
\newcommand{\nonfuncreq}[2]
{
	\subsubsection{#1}
	\begin{tabular}{|>{\raggedleft\arraybackslash}p{3cm}|p{12cm}|}
		\hline
		\textbf{Type} &
			Non - Functional
		\\
		\textbf{Description} & #2 \\
		\hline
		\if\feedbackspace1
			\textbf{Result Pass / Fail} &
				~
			\\
			\textbf{Remarks} &
				~\newline~\newline~\newline~
			\\ \hline
		\fi
	\end{tabular}
}

\section{Solution Requirements}
	\subsection{User Accounts}
		\funcreq
			{Account Registration}
			{New users must be able to create an account which is stored on 
			the server.}
			{1}{None}
			{The user will give a username, email address and password.}
			{The server will check that the username and password conform 
			to any length and composition restraints, the email is structured 
			correctly, and that the username and email address are both unique.
			}
			{If the given information is valid, a new account with the details 
			entered by the user will be stored.}
		\funcreq
			{Account Activation}
			{New accounts must have their email addresses verified with a 
			verification email. The account will not be `active' - it will not 
			provide access to restricted resources - until the user verifies 
			their email.}
			{2}{Account Registration}
			{The user will click a link on the verification email to activate 
			the account.}
			{The server will check to see if the account has already been 
			activated, or if the account has expired because it had not been 
			activated for an amount of time.}
			{The account will be marked as active if the account exists and 
			has not yet been activated. If the account is not activated 
			within a set amount of time of the email being sent, the account 
			is removed from the system.}
		\funcreq
			{Administrator Accounts}
			{It must be possible to mark an account as being owned by an 
			Administrator. These accounts will be authorised to perform more 
			actions than normal user accounts.}
			{2}{Account Registration}
			{N/A}
			{N/A}
			{An account marked as being an Administrator account will have 
			additional abilities as defined by later requirements.}
		\funcreq
			{Resend Activation Email}
			{A user must be able to request that the activation email is
			resent. This will invalidate the previously sent email.}
			{3}{Account Activation}
			{The user will supply their email address, which is where the 
			activation email will be sent.}
			{The server will check that the email address is registered to an 
			existing account that is not already active.}
			{If the email is registered to an inactivated account, an 
			activation email is sent to the user with a link to activate the 
			account.}
		\funcreq
			{User Authentication}
			{A user with an activated account must be able to authenticate 
			themselves in order to access resources restricted to account 
			owners.}
			{1}{Account Registration}
			{The user will give their username and password.}
			{The server will look for an account in the database which matches
			the username and password provided.}
			{If a matching account is found, the user is authenticated as 
			the account owner and they may access some otherwise restricted 
			parts of the system. Otherwise, the user will be informed that the
			credentials that they supplied are incorrect.}
		\funcreq
			{Account Removal}
			{An authenticated user must be able to request the deletion of 
			that account along with all personal data. This request must be 
			confirmed by the user through a confirmation email within a week.}
			{2}{User Authentication}
			{The user will need to be authenticated and also click on a link 
			in a confirmation email.}
			{The request will only be fulfilled when the user has clicked a 
			link in a confirmation email that was sent when they initiated the 
			request and it has been under a week since it was sent.}
			{When the user has confirmed the action, their account will be 
			deleted from the server along with any related personal data.}
		\funcreq
			{Administrator Account Removal}
			{A user authenticated as an administrator must be able to delete
			any non-administrator account.}
			{2}{Administrator Accounts}
			{The administrator will select which account they wish to delete.}
			{The server will check that the selected account exists and the
			user making the request is an authenticated administrator.}
			{The selected account will be removed from the system.}
		\funcreq
			{Password Reset}
			{A user should be able to request to reset their password. They do
			not need to be authenticated to do so. Upon initiating the request,
			an email will be sent to confirm the action and allow them to 
			choose a new password.}
			{3}{Account Registration}
			{The user will enter their email address.}
			{The given email address will be checked to see if it is 
			registered to a user account.}
			{If the email address does not match an account, the user is
			informed with a message. Otherwise, an email will be sent to the
			given email address containing a link to reset their password.}
		\nonfuncreq
			{Creating Accounts}
			{A user new to the application must be able to create an account in 
			5 minutes or less, excluding activation time.}
		\nonfuncreq
			{Identifying Invalid Account Creation}
			{Should the information provided by a new user when creating an
			account be invalid, they must be able to identify the exact reason
			why their input was incorrect without assistance.}
		\nonfuncreq
			{Activation Email}
			{After creating an account or requesting an activation email to be
			resent, an activation email must have been sent to them within a
			minute.}
		\nonfuncreq
			{Activating Accounts}
			{A user with an inactive account, when clicking the link on a valid
			activation email, must have their account become active within a
			minute.}
		\nonfuncreq
			{Ease of Account Creation}
			{A user with no experience of the system but with at least basic
			knowledge of operating a computer must be able to create an account
			including activation unassisted.}
		\nonfuncreq
			{Authentication Response}
			{Assuming a user has an internet connection and they give the
			username and password that corresponds to their existing account,
			they must be able to authenticate themselves in under 30 seconds
			after submitting their credentials.}
		\nonfuncreq
			{Identifying Authentication Failure Cause}
			{Should authentication fail due to incorrect credentials, a user
			new to the system should be able to identify the reason why without
			assistance.}
		\nonfuncreq
			{Account Removal Email}
			{When a user requests for their account to be removed, an email to
			confirm the action must be sent to them within a minute.}
		\nonfuncreq
			{Account Removal Confirmation}
			{Upon the confirmation of a account removal email, a user's account
			and all information associated with it must be removed within a
			minute.}
		\nonfuncreq
			{Administrator Account Removal}
			{When an administrator requests for the removal of a user's
			account, that account must be removed along with all information
			related to that account within a minute.}
		\nonfuncreq
			{Password Reset Confirmation}
			{When a user requests for their password to be reset, a
			confirmation email will be sent to them within a minute.}
		\nonfuncreq
			{Identifying Password Reset Failure Cause}
			{Should the new password entered by a user be invalid, or should
			the server be unable to update the password, a new user must be
			able to identify the reason without assistance.}
		\nonfuncreq
			{Password Reset Updating}
			{When a user has chosen a new password, the new password will
			become the active password of the account within 10 seconds.}
	\subsection{Game Mechanics}
		\funcreq
			{Account Balance}
			{Each user account must have a point balance associated with it. 
			When authenticated, the user that owns the account must be able to
			view the balance at any time.}
			{1}{Account Registration}
			{N/A}
			{The server will check that the user is authenticated.}
			{An authenticated user will be able to view their account balance.}
		\funcreq
			{Account Transactions}
			{It must be possible for points to be added and removed from a 
			user's account balance by other components of the system.}
			{1}{Account Balance}
			{The amount of points to be added or removed is specified.}
			{A withdrawal transaction will not occur if the number of points 
			to remove exceeds the number of points stored in the account.}
			{When a withdrawal or deposit occurs, the updated amount of points 
			in the account will be updated immediately.}
		\funcreq
			{Cache Ownership}
			{A cache is able to be owned by at most one user at a time. Users 
			must be able to see which caches are owned, and who owns them.}
			{1}{Account Registration}
			{A user makes a request to view the owner of a cache.}
			{The server will check to see if the cache has an owner.}
			{If the cache has an owner, the current owner will be displayed.}
		\funcreq
			{Cache Balance}
			{A cache must have a point balance associated with it, which must 
			be visible to authorised users, including the owner of the cache,
			when requested.}
			{1}{None}
			{A user makes a request to view the point balance of a cache.}
			{The server checks to see if the user is authorized to view the 
			point balance of the cache.}
			{The balance of a cache will be displayed to authorised users when 
			requested.}
		\funcreq
			{Cache Transactions}
			{It must be possible for points be added or removed from the cache
			by other components of the system.}
			{1}{Cache Balance}
			{When points are being added or removed, the amount is specified.}
			{A withdrawal transaction will not occur if the number of points 
			to remove exceeds the number of points stored in the cache.}
			{When a withdrawal or deposit occurs, the updated amount of points 
			in the cache will be visible to authorised users when requested.}
		\funcreq
			{Cache Withdrawal}
			{The owner of a cache must be able to transfer points from the 
			cache to their account when they physically visit the location of 
			the cache. An owned cache becomes unowned if the owning user 
			withdraws all points from the cache.}
			{1}{Account Balance, Cache Balance}
			{The user will specify how many points to withdraw.}
			{The server will check to see if the user performing the 
			transaction is within a given radius of the cache. The server will
			also check to ensure the point balance of the cache has at least
			the number they wish to withdraw.}
			{If the user is not within a given radius of the cache or the cache
			has less points stored in it than the requested amount to withdraw,
			the user will be informed and the transaction will not occur. 
			Otherwise, the cache and user account balances are
			updated after the transaction. If a withdrawal leaves a cache empty
			then the cache is marked as unowned.}
		\funcreq
			{Cache Depositing}
			{When they physically visit the location of a cache that is owned
			by them or does not have an owned, a user must be able to transfer 
			points to the cache from their account. After transferring one or 
			more points to an unowned cache, the unowned cache will become
			owned by the user.}
		    {1}{Account Balance, Cache Balance}
			{The user will specify how many points to deposit.}
			{The server will check to see if the user performing the 
			transaction is within a given radius of the cache. The server also
			checks to see if the user has enough points.}
			{If the user is not within a given radius of the cache or the user
			owns less points than they requested to deposit, the
			use is informed and the transaction does not occur. Otherwise the
			cache and user account balances are updated after the transaction.
			If a deposit leaves a previously empty cache populated then the
			cache is marked as owned.}
		\funcreq
			{Cache Placement Cost}
			{When placing a cache, a number of points must be deducted from the
			user's account balance.}
			{1}{Account Balance, Cache Placement}
			{A user attempts to place a cache.}
			{The server will check to see if the user has enough points in
			their account to place the cache.}
			{If the user can afford it, the cache is placed and points are 
			removed from the placing user's account.}
		\funcreq
			{Cache Scouting}
			{If a user physically visits the location of a cache owned by a 
			different user, they must be able to view the current point balance 
            of that cache.}
			{2}{Find Location, Cache Balance}
			{The location of the user is supplied.}
			{The distance of the user from the cache is checked to ensure they 
			are sufficiently near the cache.}
			{If the user is within a given radius of the cache, they will be 
			shown the number of points that is stored in the cache.}
		\funcreq
			{Cache Attacking}
			{After scouting a cache, a user must be given the option to 
			trigger an attack on that cache using points from their account.}
			{1}{Cache Scouting}
			{The attacker chooses how many points from their account they will 
			attack with.}
			{The server checks that the attacker's distance to the cache is
			less than a given radius, and has input a number of points between
			one and the number of points in their account. The server will
			decided which party survives the encounter, and how many points
			they lost in the conflict.}
			{If the request was valid, an attack is initiated on the cache by 
			that user with the specified number of points.}
		\funcreq
			{Successful Attack}
			{If the attacker wins, the ownership of the cache must pass to
			them. All defending points will be lost, and the surviving
			attacking points will transferred to the balance of the cache.}
			{1}{Cache Attacking}
			{N/A}
			{N/A}
			{The surviving points from their attack will be moved to the 
			cache's point balance, and the cache becomes owned by the attacker.
			}
		\funcreq
			{Successful Defence}
			{If the defender wins, the cache must remains theirs. All attacking 
			points are lost, and the surviving defenders remain in the cache.}
			{1}{Cache Attacking}
			{N/A}
			{N/A}
			{The surviving defenders remain in the cache, which remains owned 
			by the defending user.}
		\funcreq
			{Battle Breakdown}
			{After an attack on a cache, the attacking user will be able to
			view a breakdown of the results of the attack.}
			{3}{Cache Attacking}
			{An attack on a cache has concluded.}
			{The server will provide information such as the number of units
			lost by each side, and the reward for winning if the attacker was
			victorious.}
			{A breakdown of the results of the battle will be displayed to the
			attacking user.}
		\funcreq
			{Cache Operation Chronology}
			{All operations operations on a given cache (transactions, attacks
			and deletions) must occur in chronological order in respect to when
			they were received by the server.}
			{1}{Cache Transactions, Cache Attacking}
			{An operation on a cache performed by a user or administrator.}
			{The server will only process one request concerning operations on
			a cache at time.}
			{An operation will not be initiated until all
			previously received requests concerning operations on that cache
			have been completed.}
		\funcreq
			{Point Generation}
			{Points must be periodically supplied to each user based on their 
			current performance in the game.}
			{1}{Account Balance}
			{The point allocation system will use a user's current cache 
			number, the events of recent battles involving the user, and other 
			relevant data.}
			{The server will use the given information to decide how many
			points to give the user.}
			{Each user will periodically receive points based on their 
			performance in the game.}
		\funcreq
			{Administrator Placed Caches}
			{It must be possible for administrators to place caches without 
			needing to be at the location. These caches behave as normal, and
			the administrator can place it with as many points in them as they
			wish, including none.}
			{2}{Cache Attacking}
			{Administrators will specify the longitude and latitude of a new 
			cache to place, along with how many points will be stored there.}
			{The server will validate the location given and if the number of 
			units placed is non-negative.}
			{The cache will immediately be placed at the given location with 
			the specified number of points in its balance.}
		\funcreq
			{Non-Player Caches}
			{It must be possible for administrators to place caches which may 
			be attacked by users, but not claimed after a victory.}
			{2}{Cache Attacking}
			{Administrators will specify the longitude and latitude of a new 
			cache to place, along with how many points will be stored there.}
			{The server will validate the location given and the point balance 
			given is non-negative.}
			{The cache will immediately be placed at the given location with 
			the specified number of points in its balance.}
		\funcreq
			{Attacking Non-Player Caches}
			{A user attacking a non-player cache must receive a number of 
			points if they are victorious, but it will not become theirs. 
			After the victory, the number of points in the cache balance should 
			be reset.}
			{2}{Non-Player Caches}
			{Users may trigger an attack in the same was as they would on a 
			user controlled cache.}
			{The process for deciding the outcome of an attack on a non-player 
			cache will take the same form as one on a user cache. Each attack 
			is treated as a separate instance and each user will have to wait a
			period of time before they can attack the same non-player cache 
			again.}
			{The attacking player will receive a point reward directly to 
			their account if they are deemed to have won the battle, and the 
			point balance in the cache will be reset.}
		\funcreq
			{Scouting Non-Player Caches}
			{If a user that has attacked a non-player cache and the minimum
			delay between attacks on a non-player cache has not elapsed for
			that cache, the	user will not be allowed to scout (and by
			extension, attack) the cache.}
			{2}{Attacking Non-Player Caches}
			{A user attempts to scout a non-player cache that they attacked a
			period of time ago that is less than the minimum amount of time
			that they must wait in between attacking non-player caches.}
			{The server will check the amount of time since that user last
			attacked the cache (if at all).}
			{If the user has never attacked that cache, or the time since the
			last attack is greater than the amount of time a user must wait
			between attacks on a non-player cache, they are able to scout the
			cache as normal. Otherwise they cannot.}
		\funcreq
			{Special Event Placement}
			{It should be possible for administrators to define areas by the 
			MAC address of a nearby wireless network. This area will define a 
			collection point for a one-time-only reward which will be limited 
			to a given number of users.}
			{3}{Find MAC Address}
			{Administrators will specify the MAC address of the new cache, the 
			reward, and how many users may claim that reward.}
			{The server will validate that the given address is a valid MAC 
			address, and that the reward and number of users which can claim 
			it are greater than zero.}
			{A new special event area will be available for users to be 
			notified of and claim rewards from.}
		\funcreq
			{Special Event Rewards}
			{When a user enters the effective range of a wireless network whose 
			MAC address has been designated as a special event placement, they 
		    must be able to claim a reward from it. If it is still available,
		    they will receive points directly to their account balance.}
			{3}{Special Event Placement}
			{A user claims a reward at a given wireless network.}
			{The server will check to see if the reward is still available.}
			{If the reward is still available, a user can claim the reward. 
			The value is credited directly to their point balance and the 
			special event area is removed after the reward has been claimed by 
			a designated number of users.}
		\funcreq
			{Cache Reporting}
			{Users should have the ability to mark a cache as being placed
			unfairly. Administrators will be alerted to the reported cache.}
			{2}{Cache Placement}
			{A user will select a cache that they wish to report.}
			{The server will check that the cache is not owned by the user
			themselves, has not already been reported, and is not a non-player
			cache.}
			{If the report is valid, an administrator will be alerted to the 
			reported cache, with information including the owner, location, 
			and who reported it.}
		\funcreq
			{Administrator Cache Deletion}
			{Administrators must have the ability to delete any cache from the
			system.}
			{2}{Administrator Accounts}
			{An administrator will select a cache that they wish to remove from
			the system.}
			{The server will check that the selected cache exists and that the
			user making the request is an authenticated administrator.}
			{The selected cache will be deleted from the server and will no
			longer be visible to users.}
		\funcreq
			{Account Deletion Cache Removal}
			{When a player account is removed from the system, all caches owned
			by them must also be deleted.}
			{2}{Cache Ownership, Account Removal}
			{The process will be triggered by a user account being removed}
			{The server will find all caches owned by that account.}
			{All caches owned by the deleted player will be removed from the
			system.}
		\nonfuncreq
			{Identifying Point Balance}
			{A new user, once authenticated into an activated account, must be
			able to identify their account balance without assistance.}
		\nonfuncreq
			{Account Transactions}
			{Transactions that occur on a user's account balance must take
			effect within 5 seconds of the transaction request being received
			by the server.}
		\nonfuncreq
			{Cache Ownership Request}
			{When requested, the owner of a cache must be displayed within 20
			seconds.}
		\nonfuncreq
			{Identifying Cache Ownership}
			{A new user authenticated into an active account must be able to
			identify the owner of a displayed cache without assistance.}
		\nonfuncreq
			{Identifying Classification of Caches}
			{A new user must be able to identify what classification of cache
			(user owned, self owned, unowned or non-player) a cache is within
			10 seconds of viewing a cache on the map.}
		\nonfuncreq
			{Identifying Point Balance}
			{A new user, once authenticated into an activated account, must be
			able to identify the point balance of a cache that they are
			authorised to view without assistance.}
		\nonfuncreq
			{Point Balance Response}
			{Upon an authorised request by an authenticated user to view the
			point balance of a cache, the cache balance must be displayed
			within 20 seconds.}
		\nonfuncreq
			{Cache Transactions}
			{Transactions that occur on a cache's point balance must take
			effect within 5 seconds of the transaction request being received
			by the server.}
		\nonfuncreq
			{Identifying Cache Transaction Limits}
			{When attempting to withdrawing or deposit points at a cache that
			they own, a new user who is authenticated into an activated account
			will be able to identify how many points they can withdraw or
			deposit without assistance.}
		\nonfuncreq
			{Identifying Cache Transaction Failure Causes}
			{In all cache transactions, if a new user or administrator exceeds
			the number of points they are allowed to enter, they must be able
			to identify the reason why the transaction has not occurred without
			assistance.}
		\nonfuncreq
			{Identifying Cache Placement Cost}
			{When selecting a location for a new cache to be placed, an 
			authorised new user must be able to identify the cost of the 
			placement without assistance.}
		\nonfuncreq
			{Identifying Defending Points}
			{Upon an authorised request to scout a cache, a new user will be
			able to identify how many points are stored in the cache without
			assistance.}
		\nonfuncreq
			{Identifying Ability to Attack}
			{A new user who makes an authorised request to scout a cache will
			be able to identify how to initiate an attack on that cache without
			assistance.}
		\nonfuncreq
			{Identify Maximum Attacking Points}
			{When attempting to attack a scouted cache, a new user must be able
			to identify the maximum number of points that they can attack with
			without assistance.}
		\nonfuncreq
			{Selecting Attacking Points}
			{When attempting to attack a scouted cache, a new user must be able
			to input an exact number of points to attack with within 5
			seconds.}
		\nonfuncreq
			{Displaying Attack Result}
			{After an attack, the result of the battle must be displayed to the
			attacking user within 20 seconds.}
		\nonfuncreq
			{Updating Attack Effects}
			{After an attack, the resulting effects such as removal of
			defending points and/or transferral of ownership of the	attacked
			cache must be registered on the server within 5 seconds.}
		\nonfuncreq
			{Identifying Points Lost}
			{After attacking a cache, a new user must be able to identify how
			many points they and their opponent lost in the attack without
			assistance.}
		\nonfuncreq
			{Identifying Victory Points}
			{After attacking a cache and being victorious, a new user must be
			able to identify how many points they received without assistance.}
		\nonfuncreq
			{Administrator Cache Placement}
			{An administrator placed cache must be stored on the server ready
			to be viewed by users in under 10 seconds.}
		\nonfuncreq
			{Identifying Cache Proximity}
			{A new user should be able to identify when they are close enough
			to a cache to interact with it without assistance.}
		\nonfuncreq
			{Identifying Administrator Cache Limits}
			{A new administrator must be able to identify the minimum number of
			points they can assign to a new administrator placed cache when
			attempting to place an administrator placed cache without
			assistance.}
		\nonfuncreq
			{Identifying Attack Net Profit}
			{After attacking a cache, a new user must be able to identify their
			total point gains or losses without assistance.}
		\nonfuncreq
			{Identifying Non-Player Cache Scouting Failure Cause}
			{When a new user attempts to attack a non-player cache before the
			minimum delay between attacks since their last attack has elapsed,
			they must be able to identify the reason why the attack did not
			occur without assistance.}
		\nonfuncreq
			{Special Event Placement}
			{A special event placement definition must be stored on the server
			in under 10 seconds.}
		\nonfuncreq
			{Special Event Reward Claiming}
			{When a user claims a special event reward, the point value of the
			reward should be added to their account balance in less than 10
			seconds.}
		\nonfuncreq
			{Special Event Visibility}
			{A new user with the mobile application active must be able to
			recognise if they are in the effective range of a wireless network
			that is currently the host of a special event without assistance.}
		\nonfuncreq
			{Cache Reporting}
			{When a cache has been reported by a user, an alert must be
			visible to administrators in less than 30 seconds.}
		\nonfuncreq
			{Administrator Cache Removal}
			{When an administrator deletes a cache, the cache must be removed
			from the server within 10 seconds.}
		\nonfuncreq
			{Automatic Cache Removal}
			{When an account is removed from the server, all caches owned by
			that account must be removed within 10 seconds.}
	\subsection{Application}
		\funcreq
			{Display Location}
			{The application must allow the user to have displayed to them
			their current location in the context of a map provided by the
			Google Maps API.}
			{1}{Find Location}
			{The user will request for the Google Maps API powered map to 
			navigate to their current location.}
			{The application will check to see if the user has an Internet 
			connection and that their GPS antenna is enabled.}
			{If the user has an internet connection and their GPS antenna is 
			enabled, the application will show the user's location on a map 
			provided by the Google Maps API.}
		\funcreq
			{Nearby Caches}
			{The application must be able to show the user, on the map, the 
			locations of caches near to a specified location.}
			{1}{Find Location, Display Location, Server Connectivity}
			{The user will request to see the locations of caches near to
			a specified position.}
			{The application will check that the device's GPS antenna is 
			enabled and there is Internet connectivity. If there is a
			connection and the antenna is enabled, the application will send a
			request to the Server for a list of coordinates of caches near to
			the specified position.}
			{The application will mark on the map, provided by the Google Maps 
			API, all caches near to the specified position.}
		\funcreq
			{Map Zooming}
			{The application should allow the user to view a larger or smaller 
			area of the map.}
			{2}{Nearby Caches}
			{The user will be able to specify the `zoom level'.}
			{The application will ensure that the specified zoom level is
			within defined limits, and if so will request from the server a
			list of caches within the portion of the map that is currently
			visible.}
			{The map is displayed at different levels of zoom, as specified by 
			the user, along with any caches that should be visible at that
			level of zoom.}
		\funcreq
			{Path Finding}
			{The application should allow the user to view a path between the 
			user's current position and a target cache they specify.}
			{3}{Find Location, Display Location, Nearby Caches}
			{The user will select which cache they want to navigate to.}
			{The application will use the Google Directions API to find a path 
			to the target cache.}
			{A path will be drawn on a map to show which route to take.}
		\funcreq
			{Cache Placement}
			{The application must allow the user to request the placement of a 
			cache at their current location.}
			{1}{Find Location, Server Connectivity}
			{The user will trigger a request to place a cache at their current 
			position.}
			{The server will validate the request to place a new cache.}
			{The application will send a cache placement request to the 
			central server, and if it is successful all users will be able to 
			see the new cache.}
	\subsection{Peripheral Functionality}
		\funcreq
			{User Communication}
			{Users may be able to send messages to other users.}
			{3}{User Accounts}
			{An authenticated user will specify the message subject, and the
			message to send.}
			{The server will check that the subject and message body are not
			empty.}
			{A message will be created which is visible to all intended
			recipients.}
		\funcreq
			{Communication Reporting}
			{Users should have the ability to report communications sent
			between users as being inappropriate. Administrators will be
			alerted to the reported communication.}
			{3}{User Communication}
			{The user will specify which message to report.}
			{N/A}
			{An administrator will be alerted to the reported message,
			including information such as the sender, message content and who
			reported it.}
		\funcreq
			{Activity Recording}
			{Any game activity performed by a registered user should be
			recorded and can be viewed by that user when requested. Users will
			also be able to view activities by other users which have a direct
			effect on them.}
			{2}{User Accounts}
			{A user can request the activities that have occurred since a
			specified time and date.}
			{The server will locate all activities related to that user since
			they time they gave.}
			{Any activities that occurred since the given time and date will
			be displayed to the user.}
		\funcreq
			{Website}
			{There must be a website that is accessible by users.}
			{2}{User Accounts}
			{A user will make a request to access a certain page of the website
			through a web browser.}
			{The server will provide the requested page if it exists.}
			{A user will be able to view a requested web page if it exists.}
		\funcreq
			{Website Authentication}
			{A user must be able to authenticate themselves to their account
			through the website to be able to access restricted content on the
			site.}
			{2}{Website, User Authentication}
			{The user will give their username and password.}
			{The server will look for an account in the database which matches
			the username and password provided.}
			{If a matching account is found, the user is authenticated as 
			the account owner and they may access some otherwise restricted 
			parts of the website. Otherwise, the user will be informed that the
			credentials that they supplied are incorrect.}
		\funcreq
			{Viewing Owned Caches}
			{A user, when authenticated on the website, must be able to view a
			list of all caches that are currently owned by them, and details
			about each one, including the number of points stored inside and
			its location.}
			{2}{Website Authentication, Cache Ownership}
			{A user will request to view the web page containing a list of the
			caches they own.}
			{The server will ensure the user is authenticated.}
			{If the user is authenticated, a web page containing a list of all
			caches owned by that user's account is displayed.}
		\funcreq
			{Viewing Cache Details}
			{On the website, an authenticated user must be able to view
			information about a specified cache. This information will include
			a graphical representation of the location of that cache on a map.
			If the cache is owned by the user that made the request, further
			information such as the number of points in the cache and a history 
			of attacks made against the cache while it was owned by that user
			are also displayed.}
			{3}{Website Authentication, Cache Ownership}
			{A user will request to view a web page containing additional
			information about a specified cache.}
			{The server will check that the user is authenticated, the
			specified cache	exists, and if currently owned by the user.}
			{A web page with details about that cache will be displayed,
			including a graphical representation of the cache's location on a
			map and the current owner. If the cache is owned by the user that
			made the request, additional information will be displayed
			including the number of points in the cache and a history of its
			recent events.}
		\funcreq
			{Overview Map}
			{An authenticated user must be able to view a map, using the Google
			Maps API, which displays a subset of all caches. The user will be
			able to specify a filter to decide which caches are displayed. The
			user should also be able to use this map to select a cache to view
			information about it.}
			{2}{Viewing Cache Details}
			{A user requests to view a web page containing the overview map
			using a specified filter.}
			{The server will ensure that the user is authenticated. If they are
			authenticated, the server will find the collection of caches that
			matches the specified filter.}
			{If the user is authenticated, they will be provided with a web 
			page containing the overview map. The map will only contain the
			caches that match the given filter, and will allow the user to
			select caches on the map to access the cache details page for that
			cache.}
\renewcommand{\arraystretch}{1}
