\documentclass[a4paper]{article}

\usepackage{lipsum}
\usepackage{fullpage}
\usepackage{hyperref}

\begin{document}
	\section{Hardware and Software}
		The main product shall take the form of an ``Android app'' - a self contained program running on an Android mobile device. This is what the user will directly interact with. Our product will also require a central server in order to function. The Android app shall be able to transfer information to and from this sever.
		\subsection{Android Application}
			\subsubsection{Hardware}
				The device provided for testing the Android app is a HTC Desire X, and the application should perform well on similar devices. The relevant hardware capabilities of the Desire X are listed below\cite{htcdesirex}:

				\begin{itemize}
					\item Screen Resolution: 800 x 480 pixels
					\item Processor Speed: 1.0GHz
					\item Memory: 768MB
					\item SD Card Storage: 4GB +
					\item Battery Life: 10 - 20 hours of active use
					\item Internal GPS Antenna
				\end{itemize}

				\noindent
				The app is unlikely to require more capable hardware than what is provided by the device because any intensive processing of data can be offset to the central server. The most that the device will need to handle is some trivial real-time graphical operations, and for that the hardware is more than adequate. Memory should not be a concern because the device has a very large amount. The app should not exhaust the battery supply of the device to a prohibitive extent, and some effort should be made during development to limit the usage of battery draining resources. The product requires the ability to find its current position using a GPS system, which is an ability of the testing device.
			\subsubsection{Software}
		\subsection{Central Server}
			\subsubsection{Hardware}
				The central server program will reside on a conventional computer, and because this one machine will handle all client requests it will need much more advanced hardware than the mobile devices.

				This server shall have internet connectivity in order to communicate with the Android app clients. The server will also need to be almost constantly active, with downtime only occuring at times of the day when few clients will want to connect. The machine running the server requires a large upload bandwidth in order to service many clients in a small time frame, and this should be complemented by a high enough CPU and memory access speed to reduce the processing time of client requests. A lower bandwidth can be compensated for by reducing the size of data being sent to clients. The server should have enough free storage space to allow for expansion of the product to cover more area of the world.
			\subsubsection{Software}
	\begin{thebibliography}{9}
		\bibitem{htcdesirex}
			HTC Corporation,
			\emph{HTC Desire X Specs}.
			November 2012

			\url{http://www.htc.com/www/smartphones/htc-desire-x/#specs}
	\end{thebibliography}

\end{document}
